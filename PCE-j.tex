%%
%% 研究報告用スイッチ
%% [techrep]
%%
%% 欧文表記無しのスイッチ(etitle,eabstractは任意)
%% [noauthor]
%%

%\documentclass[submit,techrep]{ipsj}
\documentclass[submit,techrep,noauthor]{ipsj}



\usepackage[dvips]{graphicx}
\usepackage{latexsym}
\usepackage{amsmath}
\usepackage{bm}
\usepackage{amssymb}


\def\Underline{\setbox0\hbox\bgroup\let\\\endUnderline}
\def\endUnderline{\vphantom{y}\egroup\smash{\underline{\box0}}\\}
\def\|{\verb|}
%

%\setcounter{巻数}{59}%vol59=2018
%\setcounter{号数}{10}
%\setcounter{page}{1}


\begin{document}


\title{電力需要ポートフォリオ最適化問題に対する\\パウリ相関エンコーディングを用いた\\量子最適化手法の実装と評価}

\etitle{Implementation and Evaluation of a Quantum Optimization Method Using Pauli Correlation Encoding for the Power Demand Portfolio Optimization Problem}

\affiliate{TIS}{TIS株式会社 戦略技術センター}
\affiliate{Osaka}{大阪大学 大学院基礎工学研究科}
\affiliate{QIQB}{大阪大学 量子情報・量子生命研究センター}
\affiliate{Riken}{理化学研究所 量子コンピュータ研究センター}

\author{吉岡 匠哉}{Joho Taro}{TIS}[yoshioka.takuya@tis.co.jp]
\author{笹田 啓太}{Shori Hanako}{TIS}
\author{中野 裕一郎}{Gakkai Jiro}{Osaka}
\author{藤井 啓祐}{Gakkai Jiro}{Osaka, QIQB, Riken}

\begin{abstract}
  本稿では,電力需要ポートフォリオ最適化問題に対して,パウリ相関エンコーディング(PCE: Pauli Correlation Encoding)を用いた量子最適化手法の実装と評価を行う.PCEは,高次元の古典的決定変数を,少数の量子ビット上で作用する相関したPauli演算子の期待値として表現する量子符号化手法であり,物理量子ビット数を抑えたまま,多数の相関変数を同時に扱うことを可能にする.
  本研究では,実数係数を持つ密結合二次最適化問題として電力需要ポートフォリオ最適化問題を定式化し,PCEに基づく変分量子回路を設計・実装した.状態ベクトルシミュレーションにより,問題サイズ $m=20$ および $m=210$ のインスタンスに対して数値実験を行い,固定深さの変分回路における最適化の収束挙動および解の精度を評価した.
  その結果,PCEを用いることで少数の量子ビットによる効率的な問題表現が可能であり,密な二次構造を持つ電力需要ポートフォリオ最適化問題に対しても安定した最適化性能が得られることを確認した.本稿は,PCEに基づく量子最適化手法の実装上の要点と,中規模問題における基礎的性能評価を示すものである.
\end{abstract}


%
%\begin{jkeyword}
%情報処理学会論文誌ジャーナル,\LaTeX,スタイルファイル,べからず集
%\end{jkeyword}
%
%\begin{eabstract}
%This document is a guide to prepare a draft for submitting to IPSJ
%Journal, and the final camera-ready manuscript of a paper to appear in
%IPSJ Journal, using {\LaTeX} and special style files.  Since this
%document itself is produced with the style files, it will help you to
%refer its source file which is distributed with the style files.
%\end{eabstract}
%
%\begin{ekeyword}
%IPSJ Journal, \LaTeX, style files, ``Dos and Dont's'' list
%\end{ekeyword}

\maketitle

\section{はじめに}

変分量子アルゴリズム(Variational Quantum Algorithms; VQA)は,近年の量子デバイスに適した柔軟な枠組みとして,組合せ最適化問題への応用が活発に研究されている.しかしながら,VQAは本質的なスケーラビリティの課題を抱えている.すなわち,現行の量子ハードウェアで利用可能な量子ビット数は依然として限定的である一方,実用上重要な最適化問題の多くは,非常に高次元で,かつ密な二次相互作用を持つ古典的決定変数空間を対象としている.限られた量子資源と高次元な古典表現との間のこのギャップをどのように橋渡しするかは,変分量子最適化における中心的な課題である.

代表的な手法である量子近似最適化アルゴリズム(Quantum Approximate Optimization Algorithm; QAOA)では,二値決定変数を量子ビットに直接対応付ける定式化が一般的である.このような一対一対応は概念的には単純であるものの,変数数に比例して必要な量子ビット数が増加するため,大規模問題への適用は困難である.さらに,密な二次結合を持つ問題では,多数の相互作用項を実装する必要があり,回路深さの増大や複雑な結合構造を要求する点でも制約が大きい.これらの背景から,問題次元と量子ビット数とを切り離す,新たな符号化戦略の探索が強く求められている.

そのような代替的表現手法の一つとして,Sciorilliらにより提案されたパウリ相関エンコーディング(Pauli Correlation Encoding; PCE)がある.PCEでは,古典的な二値変数を個々の量子ビットに割り当てるのではなく,変分量子回路によって生成される相関したPauli演算子の期待値の符号として表現する.相関次数を $k$ に固定した場合,利用可能なPauli相関子の数は
\[
m = 3 \binom{n}{k}
\]
と組合せ的に増加し,比較的少数の量子ビット $n$ によって多数の古典変数を表現することが可能となる.特に $k=\lfloor n/2 \rfloor$ を選択した場合,相関子数は最大化される.本研究では,固定された量子ビット数に対して表現能力を最大化するため,この設定を採用する.

PCEはこれまでに,ポートフォリオ最適化問題,構造化された組合せ最適化ベンチマーク,およびウォームスタート戦略との組合せなど,いくつかの最適化設定において検討されてきた.これらの先行研究は,PCEが量子ビット効率の高い表現を提供し,多様な問題クラスにおいて良好な最適化挙動を示す可能性を持つことを示唆している.一方で,既存の検証は,中規模問題や特定の構造を持つベンチマーク問題に限定される場合が多く,実用的な問題設定への適用可能性については十分に明らかにされていない.

特に現実の最適化問題では,完全結合型の二次相互作用,実数値係数,および非一様な統計構造が現れることが多く,従来研究で扱われてきた構造化問題とは性質が大きく異なる.このような密で非構造的な条件下においてPCEがどのような性能を示すのかを明らかにすることは,その実用的・概念的有効性を評価する上で不可欠である.

本技術報告書では,電力需要ポートフォリオ最適化問題を対象として,PCEに基づく量子最適化手法の実装と評価を行う.計算モデルのサイズとして $m=20, 60, 210$ の問題を取り上げ,中規模設定における最適化挙動と実装上の特性を中心に検討する.


%1
\section{電力需要ポートフォリオ最適化問題}
\label{sec:power_portfolio}

電力需要ポートフォリオ最適化問題は,電力アグリゲータの運用において発生する重要な意思決定問題であり,需要家の部分集合を選択することで,調達目標を満たしつつ,需要の不確実性や変動性を管理することを目的とする~\cite{Albadi2008,Siano2014,Palensky2011}.この種の問題は,自然に大規模な組合せ最適化問題として定式化されることが多く~\cite{CRIEPI_C18005,Faia2021,Ajagekar2019,YoshiokaFQAOA2024},需要家間の需要プロファイルの相関を反映した密な二次相互作用を含む点に特徴がある.その結果として,本問題は実数値係数を持つ高次元かつ完全結合型のQUBO(Quadratic Unconstrained Binary Optimization)問題の代表例となる.本研究で扱う電力需要ポートフォリオ最適化問題は,第\ref{sec:introduction}節で述べた密な二次最適化問題の具体的な実問題インスタンスである.

\subsection{問題設定}
\label{subsec:problem_setting}

本研究では,不確実性下で所定量の電力を調達するために,電力アグリゲータが需要側リソースの部分集合を選択する電力需要ポートフォリオ最適化問題を考える~\cite{CRIEPI_C18005,YoshiokaFQAOA2024}.需要家 $i$ の時刻 $t$ における電力消費量を $p_{t,i}$ とし,需要の確率的変動を反映する確率変数として扱う.アグリゲータは,ポートフォリオに含まれる需要家の総需要が,あらかじめ与えられた調達目標 $P^{\mathrm{target}}_t$ にできるだけ一致するようにしつつ,需要の不確実性に起因する変動を抑制することを目的とする.

ポートフォリオの選択は,二値決定変数
${\bm x}=(x_1,\ldots,x_m)\in\{0,1\}^m$
によって表され,$x_i=1$ は需要家 $i$ をポートフォリオに含めることを意味する.この定式化は,二次のコスト構造を持つ高次元の組合せ最適化問題を自然に導き,大規模な設定では古典的な厳密解法にとって困難な問題となる.

本研究で扱う問題サイズは,パウリ相関エンコーディングによって可能となるスケーリング特性を反映するように選定されており,Fig.~\ref{fig:curve_compression} に示す圧縮曲線上の代表的な点をカバーする.具体的には,詳細な挙動解析が可能な小規模問題から,限られた量子ビット数で表現可能な上限に近い中規模問題までを対象とする.

\subsection{パウリ相関エンコーディング}

パウリ相関エンコーディング(Pauli Correlation Encoding; PCE)は,多数の決定変数を相関したPauli演算子の期待値として埋め込むことで,大規模な組合せ最適化問題を効率的に表現する手法である.Fig.~\ref{fig:curve_compression} に示すように,PCEでは量子ビット数 $n$ に対して有効な問題次元 $m$ が組合せ的に増大するため,比較的少数の量子ビットを用いて,現実的な規模の電力需要ポートフォリオ問題を扱うことが可能となる.

特に,相関次数を $k=n/2$ と選択することで,異なるPauli相関子の数が最大化され,与えられた量子ビット数に対して最大の有効問題サイズ $m_{\max}$ が得られる.同時に,この選択は対称性の高い相関構造をもたらし,安定した最適化挙動に寄与する.本研究では,この設定を一貫して採用する.

\subsubsection{本研究で用いる変分量子回路}

本研究では,量子回路を単独の量子アルゴリズムとしてではなく,Pauli相関子を生成する学習型のパラメータ付き量子回路として用いる.回路の深さは全ての計算において5に固定する.ここで1層は,
(i) 全量子ビットに対する $R_y$ 回転,
(ii) 全量子ビットに対する $R_z$ 回転,
(iii) 全結合型の $ZZ$ 相互作用
の逐次適用から構成される.

回路深さを $l$ とした場合,各層には $\binom{n}{2}$ 個の二量子ビット $ZZ$ 相互作用と,$O(n)$ 個の単一量子ビット回転が含まれ,全体として $O(l\,n^2)$ 個のエンタングリングゲートを要する.本実装では,各層あたり $\binom{n}{2}+2n$ 個の基本ゲートに相当する.なお,回路深さを増加させた予備的検証においても,収束挙動に本質的な変化は見られず,最適化性能を支配する主要因は回路深さよりもPauli相関子による表現空間の大きさであることが示唆された.

\subsubsection{コストモデル:時間平均型と時間分解型}
\label{subsubsec:cost_models}

本研究では,電力需要ポートフォリオ最適化問題に対して,時間分解能の異なる2つの関連したコストモデルを導入する.両モデルはいずれも同一の時間別コスト関数に基づいているが,需要変動の扱い方が異なる.以下では,ポートフォリオの次元,すなわち資産数を $m$ とする.

\paragraph{時間別コスト関数}

時刻 $t$ におけるポートフォリオ構成 ${\bm x}$ に対する時間別コストを
\begin{align}
  C_t({\bm x})
  =&
  \frac{1}{m^2}
  \mathbb{E}\!\left[
    \left(
      \sum_{i=1}^{m} p_{t,i} x_i - P^{\mathrm{target}}_t
    \right)^2
  \right] \nonumber\\
  =&
  \frac{1}{m^2}
  \sum_{i=1}^{m}\sum_{j=1}^{m}
  \mathrm{Cov}(p_{t,i},p_{t,j}) x_i x_j \nonumber\\
  &+
  \frac{1}{m^2}
  \left(
    \sum_{i=1}^{m}\mathbb{E}[p_{t,i}] x_i
    - P^{\mathrm{target}}_t
  \right)^2
  \label{eq:hourly_cost}
\end{align}
と定義する.ここで $m^2$ による正規化は,ポートフォリオサイズの増大に伴う目的関数値および勾配の過度な増加を抑え,変分最適化の数値安定性を向上させるために導入している.この量は,各時刻における調達電力量と目標値との差を,需要の確率変動と平均値の双方を考慮した形で評価するものである.

\paragraph{データセットと問題生成}

次に,本研究で用いるデータセットおよび異なる規模の問題インスタンスの生成方法について述べる.モデルパラメータは,Energy Management System Open Data プラットフォームで公開されている電力消費データに基づいて構成した~\cite{ems_opendata}.データセットは,4月1日から5月30日までの61日間にわたる1時間毎の電力消費記録からなり,$2{,}143$ 個の独立な需要家時系列を含む.

各需要家 $i$ および時刻 $t$ に対して,利用可能な需要削減量 $p_{t,i}$ を元の電力消費量の10\%として定義する.また,時刻 $t$ における調達目標電力量は,全需要削減量の半分として
\[
P^{\mathrm{target}}_t = \frac{1}{2}\sum_{i=1}^{m} p_{t,i}
\]
と設定する.

$m \le 2{,}143$ の問題では,利用可能な需要家集合から異なる $m$ 人を無作為に選択してポートフォリオを構成する.一方,$m > 2{,}143$ の場合には,mixup データ拡張法を用いて追加の需要家プロファイルを生成する.この手法では,独立な2つの需要家時系列の凸結合として合成需要プロファイルを構成する.この手続きにより,元データの統計的性質を保ちつつ,幅広いポートフォリオ規模に対する体系的な検証が可能となる.

\paragraph{モデル1:時間平均型ポートフォリオ最適化}
\label{par:model1}

モデル1は,長さ $n_T$ の時間窓にわたって共通のポートフォリオを選択する時間平均型の定式化である.このモデルは,需要不確実性の支配的な統計構造を捉えることを目的として導入され,量子最適化における有効な初期回路パラメータを得るために用いられる.

時間平均コスト関数は
\begin{align}
  C_T({\bm x})
  &=
  \frac{1}{n_T}
  \sum_{t=T}^{T+n_T}
  C_t({\bm x})
\end{align}
と定義される.$C_t({\bm x})$ の定義を代入することで,
\begin{align}
  C_T({\bm x})
  &=
  \frac{1}{m^2}\sum_{i=1}^{m}\sum_{j=1}^{m}
  \overline{\mathrm{Cov}(p_{t,i},p_{t,j})}
  x_i x_j
  \nonumber \\
  &\quad
  + \frac{1}{n_T}
  \sum_{t=T}^{T+n_T}\frac{1}{m^2}
  \left(
    \sum_{i=1}^{m}\mathbb{E}[p_{t,i}] x_i
    - P^{\mathrm{target}}_t
  \right)^2
  \label{eq:averaged_cost}
\end{align}
と書き換えられる.ここで上線は時間窓 $n_T$ にわたる平均を表す.本研究では基準時刻を $T=1$ とし,平均化窓は $n_T=24$ に固定する.この平均化モデルは,粗視化された最適化ランドスケープを与え,初期パラメータ決定に特に適している.

\paragraph{モデル2:時間分解型ポートフォリオ最適化}
\label{par:model2}

モデル2は,時間ごとのコストを積算して最小化する完全な時間分解型の定式化であり,実運用における需要変動下での調達性能評価を表すモデルである.

後の解析のために,各時刻 $t$ における期待調達電力量およびその変動幅を定義する.期待調達電力量は
\begin{align}
P_t(\bm{x}) = \sum_{i=1}^{m} \mathbb{E}[p_{t,i}] x_i
\end{align}
で与えられ,対応する標準偏差は
\begin{align}
\sigma_t(\bm{x}) =
\sqrt{
\sum_{i=1}^{m}\sum_{j=1}^{m}
\mathrm{Cov}(p_{t,i},p_{t,j}) x_i x_j
}
\end{align}
と定義される.これらは調達電力量の平均および変動性を特徴付ける量であり,
Sec.~\ref{sec:numerical_results} におけるポートフォリオの頑健性評価に用いられる.これらを用いると,
Eq.~(\ref{eq:hourly_cost}) の時間別コストは
\begin{equation}
C_t(\bm{x}) =
\frac{1}{m^2}\sigma_t(\bm{x})^2
+ \frac{1}{m^2}\left[
P_t(\bm{x}) - P^{\mathrm{target}}_t
\right]^2
\label{eq:cost_decomposition}
\end{equation}
と直感的に表現できる.

最適化結果の正規化および比較のために,基準となる最小・最大コスト値を定義する.最小コスト $C_t^{\min}$ および $C_T^{\min}$ は,相対最適性ギャップ $10^{-4}$ を許容した Gurobi オプティマイザによって求める.対応する最大値 $C_t^{\max}$ および $C_T^{\max}$ は,同一の問題インスタンス内で一貫した方法により定義する.これらの値は,本研究全体を通じて正規化コストおよび損失指標を定義するための参照値として用いる.

\subsubsection{連続緩和と損失関数}
\label{subsubsec:loss_function}

変分量子最適化を可能にするため,本研究では二値変数 ${\bm x}$ を,
量子回路のパラメータ ${\bm\theta}$ によって定まる連続変数
${\bm y}({\bm\theta})\in[0,1]^m$ へと緩和する.
変分パラメータ ${\bm\theta}$ は,量子回路をシミュレーションし,
必要なPauli演算子の期待値を評価することで得られる損失関数を最小化するように最適化される.
本研究におけるすべての変分最適化は,
Qulacs シミュレータ~\cite{qulacs} を用いたノイズレスな状態ベクトルシミュレーションにより実行した.
得られる損失関数は滑らかな形状を持つため,
勾配に基づく古典最適化手法を適用することが可能である.
本研究では,検討したすべての問題サイズにおいて
安定かつ効率的な収束が確認された
Broyden--Fletcher--Goldfarb--Shanno(BFGS)法を用いて最適化を行った.
損失関数の勾配は期待値から厳密に計算され,
数値効率向上のため,状態ベクトルシミュレータに実装された
バックプロパゲーション機能を用いて評価した~\cite{qulacs}.

PCEの枠組みにおいて,緩和変数は $k$ 体のPauli相関演算子 $\Pi_i^{(k)}$ の期待値を用いて
\begin{align}
  y_i({\bm\theta})
  &=
  \varsigma\!\left(
    2\alpha
    \langle\psi({\bm\theta})|\Pi_i^{(k)}|\psi({\bm\theta})\rangle
  \right)
\end{align}
と定義される.
ここで $\varsigma(x)=[1+e^{-x}]^{-1}=[\tanh(x/2)+1]/2$ はシグモイド関数であり,
$\alpha$ は緩和の鋭さを制御するパラメータである.

これらの緩和変数を用いて,時間分解型モデルに対する損失関数を
\begin{align}
  L_t({\bm\theta})
  =
  C_t({\bm y}({\bm\theta}))
  +
  \frac{\beta\nu}{m}
  \sum_{i=1}^{m}
  \left[
    y_i({\bm\theta})-\frac{1}{2}
  \right]^2
\end{align}
と定義する.
ここで $\beta$ は正則化の強さを制御する調整パラメータであり,
$\nu=\|Q\|_F$ はQUBO行列のフロベニウスノルムを表す.
時間平均型モデルに対しても,同様に $C_t$ を $C_T$ に置き換えることで損失関数を定義する.
$\nu$ の評価においては,
${(P^{\mathrm{target}}_t})^2$ のように決定変数に依存しない定数項は除外している.

量子ハードウェア実験は,変分最適化が十分に収束した後にのみ実施する.
具体的には,シミュレーションにより得られた最適パラメータを固定した状態で量子回路を実行し,
対応するコスト値を測定する.

\subsubsection{デコーディングおよびウォームスタート戦略}
\label{subsubsec:warmstart_decoding}

実際の計算では,時間平均型モデル(モデル1)で最適化された回路パラメータを,
時間分解型モデル(モデル2)の初期値として用いるウォームスタート戦略を採用する.
この方法により,実運用を想定した定式化においても,
収束性および最適化の安定性が大きく向上することを確認した.

最適化終了後,Pauli相関演算子の期待値を用いて,
二値のポートフォリオ選択を以下のようにデコードする:
\begin{align}
  x_i
  :=
  \frac{1}{2}
  \left\{
    \mathrm{sign}
    \left[
      \langle\psi({\bm\theta})|\Pi_i^{(k)}|\psi({\bm\theta})\rangle
    \right]
    +1
  \right\}.
\end{align}
得られた二値解は,元のコスト関数を用いて再評価され,
調達精度および安定性の観点から性能評価を行う.

さらに,量子回路の変分パラメータが完全に最適化され,
最適化過程が終了した後,
最適化軌道全体の中で最小のコスト値を与えた解に対して
後処理を適用する.
この後処理として,
Sciorilliらにより提案された局所探索法~\cite{Sciorilli2025PCE} と
貪欲法の双方を検証した結果,
貪欲法の方が一貫して良好な性能を示すことが確認された.
そのため,本研究では貪欲法を後処理手法として採用する.


\section{数値結果}
\label{sec:numerical_results}

本節では,電力需要ポートフォリオ最適化問題に対する
パウリ相関エンコーディング(PCE)を用いた量子最適化手法の
数値的性能を評価する.
計算モデルのサイズとして
$m=20, 60, 210$ の3種類を取り上げ,
時間平均型モデル(Model 1)および時間分解型モデル(Model 2)
の双方について検証を行う.

まず,階層的最適化枠組みの初期段階として
時間平均型最適化問題(Model 1)を評価する.
Model 1 を解くことで,
コストランドスケープの支配的構造を反映した
変分回路パラメータが得られ,
これを時間分解型最適化(Model 2)の初期値として利用する.

\subsection{PCEに基づく最適化のスケーラビリティ(Model 1)}

Fig.~\ref{fig:cost_convergence} に,
Model 1 における問題サイズ $m$ と正規化コストの関係を示す.
最適化設定および収束統計を
Table~\ref{tab:optimization_results} にまとめる.

$m=20, 60, 210$ のいずれの問題サイズにおいても,
PCEに基づくソルバは低い正規化コストを安定して達成しており,
少数の量子ビットによる効率的な問題表現と
安定した最適化挙動が確認できる.
また,一部のサイズについて実施した
量子プロセッサ(QPU)による実行結果も,
シミュレーション結果と整合的な傾向を示した.

\begin{table}[t]
  \centering
  \caption{
    Model 1 における最適化設定および収束統計の概要.
    $m$ は需要家数(問題サイズ),
    $n$ はPCEで用いた量子ビット数,
    $k$ は相関次数を表す.
    $n_{\mathrm{params}}$ は変分回路パラメータ数,
    $n_{\mathrm{runs}}$ は異なる乱数初期値による独立試行回数,
    $n_{\mathrm{iter}}$ は最小損失を達成した試行における反復回数である.
  }
  \label{tab:optimization_results}
%  \begin{ruledtabular}
    \begin{tabular}{rrrrrr}
      $m$ & $n$ & $k$ & $n_{\mathrm{params}}$ &
      $n_{\mathrm{runs}}$ & $n_{\mathrm{iter}}$ \\
      \hline
      20  & 4 & 2 & 70  & 10 & 200 \\
      60  & 6 & 3 & 135 & 10 & 500 \\
      210 & 8 & 4 & 220 & 10 & 1300 \\
    \end{tabular}
%  \end{ruledtabular}
\end{table}

\subsection{量子ハードウェアによる検証}

シミュレーション結果に加えて,
ノイズレスシミュレーションで最適化された回路パラメータを用い,
量子プロセッサ上での概念実証実験を実施した.
生の量子出力はノイズの影響により性能劣化が見られるものの,
単一ステップの貪欲後処理を適用することで,
コスト値はシミュレーション結果に近い水準まで改善された.
この結果は,
PCEに基づく回路構成が実機上でも一定の頑健性を持つことを示唆している.

\subsection{時間分解型最適化とウォームスタート効果(Model 2)}


次に,時間分解型モデル(Model 2)に対する最適化挙動を評価する.
Fig.~\ref{fig:ite_process_warmstart} に,
$m=210$ の場合の収束過程を示す.

ランダム初期化に比べ,
時間平均型モデルで得られたパラメータを用いた
ウォームスタート初期化は,
収束速度の向上および最終到達コストの低減に寄与することが確認された.
この結果は,
階層的最適化戦略が,
実運用を想定した時間分解型問題に対して有効であることを示している.

以上の結果より,
PCEを用いた量子最適化手法は,
$m=20, 60, 210$ の範囲において
安定した最適化性能を示し,
電力需要ポートフォリオ最適化問題に対する
有効な量子表現および最適化フレームワークとなり得ることが確認された.

\section{まとめ}
\label{sec:conclusion}
本稿では,パウリ相関エンコーディング(Pauli Correlation Encoding; PCE)に基づく
変分量子最適化手法を用いて,
電力需要ポートフォリオ最適化問題に対する実装と評価を行った.
高次元の決定変数を相関した量子演算子へと圧縮することで,
量子ビット数に対して有効な問題サイズが組合せ的に増大する
スケーラブルな表現が可能となることを示した.
この関係は Fig.~\ref{fig:curve_compression} に要約されている.

数値実験では,
計算モデルのサイズとして $m=20, 60, 210$ を取り上げ,
需要の不確実性を考慮した状況下においても
安定した調達性能と良好な収束挙動が得られることを確認した.
また,回路シミュレーションと量子ハードウェア実行を組み合わせた評価により,
現実的なハードウェア制約下においても,
本手法が実行可能であることを示した.

本研究は,量子ネイティブな最適化手法と,
エネルギーシステム分野における実問題との間に
具体的な接点を与えるものであり,
量子情報科学と電力工学の融合領域における
今後の研究に向けた基盤を提供する.
本稿では電力需要ポートフォリオ最適化問題を対象としたが,
提案した枠組みはエネルギーシステムに限らず,
実数値係数を持つ一般の密な二次最適化問題にも
直接適用可能である.

\section*{謝辞}

本研究の一部は,文部科学省「量子・AI融合技術ビジネス開発グローバル拠点」(Q-LEAP)
(課題番号:JPMXS0118067394,JPMXS0120319794)の支援を受けて実施された.
また,本研究は,科学技術振興機構(JST)
共創の場形成支援プログラム(COI-NEXT)
(課題番号:JPMJPF2014)の支援を受けている.

\begin{thebibliography}{99}
  
\bibitem{vqa}
Cerezo, M. et al.,
Variational Quantum Algorithms,
Nature Reviews Physics (2021).

\bibitem{markowitz}
Markowitz, H.,
Portfolio Selection,
Journal of Finance (1952).

\end{thebibliography}

\end{document}

